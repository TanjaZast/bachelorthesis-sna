%*****************************************
\chapter{Fazit und Ausblick}\label{ch:fazit}
%*****************************************

Nachdem wir uns in dieser Arbeit ausführlich mit der  Generierung und Analyse sozialer Netzwerk beschäftigt haben, wollen wir nun die wichtigsten Erkenntnisse zusammenführen. Die Analyse sozialer Netzwerke besteht aus vielen Faktoren. Es gibt zahlreiche Methoden um eine Analyse durchzuführen und viele charakteristische Merkmale, die bei dieser von Bedeutung sind. Wir konnten in dieser Arbeit zeigen, dass die Betrachtung der Cliquen und Brücken bei der visuellen Interpretation durchaus ausreichen. Bei der Analyse der Daten wurde deutlich, dass die ausschließliche Betrachtung der einzelnen Zentralitäten bei großen Datensätzen nicht optimal ist und viel Nacharbeit in Anspruch nimmt, um die wichtigsten Knoten herauszufiltern und keine hunderttausende Werte zu vergleichen. Vielmehr ist es aussagekräftig in solchen Fällen die Verteilungen zu betrachten. Doch gleichzeitig müssen wir beachten, dass soziale Netzwerke unterschiedlichste Thematiken darstellen. Alleine eine kurze Recherche im Internet präsentiert unzählige unterschiedliche Netzwerke. Doch haben diese eine Gemeinsamkeit, sie sind alle typische soziale Netzwerke. Bei dem bloßen Vergleich von Zentralitäten sind die visuellen Differenzen oder Unstimmigkeiten zunächst nicht von Bedeutung, doch spätestens bei der Analyse der Verteilungen wird diese bedeutsam. Wir konnten jedoch in dieser Arbeit sehen, dass wir es durch Optimierungen im Code schaffen, leicht visuell ähnliche Graphen zu erzeugen. Sobald die Graphen gleiche Grundstrukturen aufweisen, bestehen starke Gleichheiten in der Verteilung der Zentralitäten. Diese Optimierung kann unendlich lange fortgeführt werden, doch erreichen wir so lediglich Kopien von existierenden Netzwerken und wertschätzen die Individualität dieser nicht. Diese Arbeit lässt einige Punkte offen, die durchaus weiter optimiert werden können. Beispielsweise die generierten Plots noch besser an existierende Graphen anpassen, um die Verteilung bestmöglich nachzustellen. Zudem größere Datensätze untersuchen und vergleichen, was sich bei Millionen von Knoten an den Zentralitäten, der Verteilungen dieser, Cliquen und Brücken verändert. Vor allem zu untersuchen, ob womöglich Regelmäßigkeiten auftreten. Oder ebenfalls interessant, ob die Berechnungen der Zentralitäten bereits optimierte Algorithmen sind und ob nicht möglicherweise doch Optimierungspotenzial besteht. Ebenfalls interessant statt der Analyse eine Interpretation der Graphen durchzuführen. Bei gegebenen Graphen, ohne Vorwissen über die Datensätze, Vermutungen aufzustellen und diese auszuführen. Dies und vieles mehr sind weitere Eigenschaften, die untersucht werden können. Was uns wieder zeigt, wie vielfältig die soziale Netzwerkanalyse ist und warum sie zahlreiche Wisschenschaftler*innen beschäftigt. Dank ihr konnten alleine im geschichtlichen Aspekt des Menschen viele Fragen geklärt oder zumindest Vermutungen aufgestellt und belegt werden. Doch nicht nur in der Vergangenheit spielt die Analyse eine große Rolle, im Bereich des Sozial Media Booms ist sie aktuell und wird auch garantiert in der Zukunft vieles beeinflussen. Seien es Unternehmen, die dadurch bekannte \textit{Influencer} finden können oder App-Entwickler, die Korrelationen zwischen Nutzer*innen auf diese Weise darstellen und Apps weiter anpassen, um ihre Nutzeranzahl zu erhöhen. Schließlich können wir diese Arbeit damit beenden, dass es unglaublich vielzählige Methoden zu Analyse von Netzwerken gibt und jede Vor- und Nachteile aufweist. Welche die geeignetste ist, lässt sich nicht in einem Satz formulieren. Es kommt auf Anzahlen von Kanten und Knoten an, aber auch auf die zu untersuchende Thematik. Optimaler weise ist es stets ratsam mehrere Methoden zu verwenden, denn diese unterscheiden sich ebenfalls in ihrer Aussagestärke.