%*****************************************
\chapter{Die Umsetzung}\label{ch:examples}
%*****************************************
\section{Vergleich mit Twitter}
Da die vorherigen Interpretationen immer ziemlich einseitig waren und die Ergebnisse klar einzuordnen waren gehen wir nun etwas tiefer in die Materie. Es gibt viele Studien und Untersuchungen zu Twitter. Vor allem auf Ebenen der sozialen Netzwerkanalyse sind diese vielversprechend und interessant. Im vorherigen Teil damit beschäftigt haben, wie soziale Netzwerke so gut und realitätsnah wie möglich konstruiert werden könne. Daher möchten wir nun die Werte, welche in dieser Arbeit generiert wurden mit den Werten, die bei der sozialen Netzwerkanalyse von Twitter berechnet wurden, verglichen werden. Leitfragen sind hierbei, was zu erwarten ist, ob die Ergebnisse den Erwartungen entsprechen oder komplett widersprechen und warum dies der Fall ist. Zusätzlich vielleicht auch Möglichkeiten erarbeitet, wie die Graphen bzw. die Generierung angepasst werden könnte um noch bessere Graphen zu erhalten, die sozialen Netzwerken noch mehr ähneln. 

\section{Twitter SNA - Vorarbeit}
Zunächst stellt sich hier die Frage welche Datensätze benutzt werden sollten. Da zu diesem Zeitpunkt nicht klar war, ob die Datensätze zwar verwendet werden dürfen aber jegliche Berechnungen selbständig durchgeführt werden müssen, bin ich nun nach meiner Vorstellung vorgegangen. Auf meiner Suche bin ich auf mehrere gute wissenschaftliche Arbeiten gestoßen. 
Zunächst müssen wir eine vergleichbare Basis schaffen. Dies schaffen wir dadurch, dass wir ungefähr die gleiche Anzahl an Knoten haben damit die Werte ansatzweise vergleichen zu können.