%************************************************
\chapter{Einleitung}\label{ch:einleitung}
%************************************************
Der Begriff $"$Soziales Netzwerk$"$ oder auf Englisch $"$Social Network$"$ weckt seit vielen Jahrzehnten das Interesse zahlreicher Sozial- und Verhaltenswissenschaftler*innen. Auch weckt es das Interesse von Unzähligen Unternehmen, um gezielter auf das Kundenverhalten einzugehen und dadurch den Gewinn zu maximieren. Doch nicht zu vergessen sind es heutzutage letztendlich die Nutzer*innen der Social Media-Plattformen wie Twitter, Facebook und Instagram, welche dieser Begriff vor allem tangiert und die Liste könnte noch lange weitergeführt werden.\\
Jedoch spezialisieren sich vor allem Sozial- und Verhaltenswissenschaftler*innen, ebenso Unternehmen, auf die Analyse sozialer Netzwerke. Diese fokussieren sich weitestgehend auf Beziehungen zwischen sozialen Einheiten, sowie die Muster und Implikationen, welche diesen Beziehungen zugeschrieben werden.
Schnell kommen Fragen auf wie, was ist ein $"$Soziales Netzwerk$"$ definiert. Oder wie eine solche Analyse aussehen kann. Was jede einzelne Methode zur Analyse auszeichnet und Welche davon als besonders vielversprechend gelten.


\section{Zielsetzung}\label{sec:zielsetzung}
\marginpar[Note 1: text for left-hand side text]{\tiny Note 1: Hier am Ende abgleichen mit der kompletten Arbeit, ob dies auch wirklich das Ziel war.}
Um eine Aussage darüber treffen zu können, welche Methoden zur Analyse geeignet sind und nicht, muss zunächst ein Verständnis für soziale Netzwerke und anschließende Analyse hergestellt werden. Diese Arbeit wird daher in zwei Bereiche unterteilt. Zum Einen beginnt sie mit der Einführung in die sozialen Netzwerke und die Einarbeitung in die verschiedenen Zentralitäten, die bei der Analyse verwendet werden. Diese geben einen guten Aufschluss darüber, wie die Einheiten miteinander verbunden sind beziehungsweise zusammenhängen. Ob es sich starke Verbindungen oder schwache handelt. Danach wird eine weitere Methode vorgestellt, welches es durch Zuordnung der Zentralitäten ermöglicht, die mathematische Gaußverteilung nachzustellen. Anhand dieser sind dann weitere Aussagen über den Graphen möglich. Anschließend wird im zweiten Teil dieser Arbeit ein Generator programmiert, welcher Soziale Netzwerke so gut wie möglich nachstellt. Um aber bewerten können, ob dieses Netzwerk eine gute Simulation ist, wenden wir die im ersten Teil der Arbeit vorgestellten Methoden an. Ziel der Arbeit ist es daher, ein gutes Verständnis für die soziale Netzwerkanalyse zu bekommen und für beliebige Netzwerke, durch Anwendung der kennengelernten Methoden, gute Bewertungen oder Analysen durchzuführen.
Diese Arbeit distanziert sich von dem Begriff $"$Social Networking$"$, welcher bei Recherchen zahlreichst auftaucht, aber lediglich den Vorgang oder Zustand beschreibt, dass Menschen über soziale Netzwerke durch beispielsweise gemeinsame Interesse zueinanderfinden. 




%*****************************************
%*****************************************
%*****************************************
%*****************************************
%*****************************************
