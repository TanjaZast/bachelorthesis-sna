%*****************************************
\chapter{Fazit und Ausblick}\label{ch:fazit}
%*****************************************

Nachdem in dieser Arbeit ausführlich die Generierung und Analyse sozialer Netzwerk behandelt wurde, werden nun die wichtigsten Erkenntnisse zusammengeführt. Die Analyse sozialer Netzwerke besteht aus vielen Faktoren. Es gibt zahlreiche Methoden um eine Analyse durchzuführen und viele charakteristische Merkmale, die bei dieser von Bedeutung sind. In dieser Arbeit wurde lediglich ein Teil davon betrachtet. Zentralitäten spielen bei der Analyse eine wichtige Rolle und in direktem Zusammenhang mit Cliquen und Brücken stehen. Die Arbeit hat gezeigt, dass Cliquen höhere \textit{Grad-Zentralitäten} aufweisen, und hohe \textit{Zwischen-Zentralitäten} bedeuten, dass dieser Knoten relevant für Brücken zwischen den Teilgraphen ist. In dieser Arbeit wurde gezeigt, dass die Betrachtung der Cliquen und Brücken bei der visuellen Interpretation sehr aufschlussreich ist und bereits Vermutungen entstehen lässt. Bei der Analyse der Daten ist im Laufe dieser Arbeit deutlich geworden, dass es hilfreich ist, die Verteilungen dieser zu betrachten. Denn oftmals werden Graphen analysiert mit einer großen Menge an Knoten. Hier bietet die Betrachtung der Verteilung eine gute Möglichkeit, einen Überblick der Zentralitäten zu bekommen und parallel, ohne den Plot dazu gesehen zu haben, die Visualisierung zu erahnen. Gleichzeitig ist zu beachten, dass soziale Netzwerke unterschiedlichste Thematiken darstellen. Alleine eine kurze Recherche im Internet präsentiert unzählige unterschiedliche Netzwerke. Doch haben diese eine Gemeinsamkeit, sie sind alle auf ihre Weise soziale Netzwerke und erfüllen dennoch Eigenschaften wie mehrere Teilgraphen, Cliquen, Brücken und ähnliche Verteilungen von Zentralitäten und teilweise dennoch visuelle Ähnlichkeiten. Zwar ist die Interpretation dieser Graphen, abhängig von der dargestellten Thematik sehr unterschiedlich, doch die untersuchten Merkmale können dennoch ähnliche Ergebnisse erzielen. Anhand dieser Arbeit wurde ebenfalls ersichtlich, dass bereits kleine Optimierungen im Quelltext des Generators, die Generierung visuell ähnlicher Graphen ermöglicht. Sobald die Graphen gleiche visuelle Grundstrukturen aufweisen, folgen diesen auch starke Gleichheiten in der Verteilung der Zentralitäten. Diese Arbeit lässt einige Punkte offen, die durchaus noch weiter optimiert werden können. Beispielsweise die generierten Plots noch besser an existierende Graphen anpassen, um die Verteilung bestmöglich nachzustellen. 
Der Generator in dieser Arbeit bildet lediglich mehrere Subgraphen und verbindet diese miteinander, wobei es in \textit{sozialen Netzwerken} auch Knoten geben kann, die sich zwischen den Subgraphen befinden, siehe Abbildung \ref{fig:FacebookGraph}. Dies könnte durch eine weitere Methode im Quelltext ermöglicht werden. Zudem wäre ein weiterer interessanter Faktor die Dichte in diesen Subgraphen, ob die Knoten sehr nah beieinander liegen oder weit voneinander entfernt sind. Dies würde sich wiederum auf die Werte der Zentralitäten und schließlich deren Verteilung auswirken.
Auch größere Datensätze, als in dieser Arbeit, zu untersuchen und zu vergleichen wäre eine interessante Fortsetzung dieser Arbeit. Ein weiterer Ansatz für die Fortestzung der Arbeit wäre die Untersuchung, ob die Algorithmen zur Berechnungen der Zentralitäten bereits optimiert sind und ob nicht möglicherweise doch Verbesserungspotenzial besteht.  
Alle diese Ideen zeigen erneut, wie vielfältig soziale Netzwerke und die Analyse dieser sind und warum sie zahlreiche Wisschenschaftler*innen seit Jahren beschäftigt. Schließlich kann diese Arbeit damit beendet werden, dass es unglaublich vielzählige Methoden zu Analyse von Netzwerken gibt. Welche die geeignetste ist, ob es möglicherweise viel bessere gibt, die wir nicht betrachtet haben, lässt sich nicht in einem Satz zufriedenstellend beantworten. Es kommt auf die Anzahl der Kanten und Knoten an, aber auch auf die zu untersuchende Thematik.
Weitere Plots, Daten, Quelltext und Ideen befinden sich im Git Repo \cite{TZ}, falls diese Arbeit das Interesse geweckt hat.