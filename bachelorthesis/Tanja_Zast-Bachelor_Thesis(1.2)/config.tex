%----------------------------------------------------------------------------------------
%	CHAR ENCODING
%----------------------------------------------------------------------------------------
% 0. Set the encoding of your files. UTF-8 is the only sensible encoding nowadays. If you
% can't read äöüßáéçèê∂åëæƒÏ€ then change the encoding setting in your editor, not the line 
% below. If your editor does not support utf8 use another editor!

\PassOptionsToPackage{utf8}{inputenc}
  \usepackage{inputenc}

\PassOptionsToPackage{T1}{fontenc} % T2A for cyrillics
  \usepackage{fontenc}


%----------------------------------------------------------------------------------------
%	CONFIGURATION
%----------------------------------------------------------------------------------------
% 1. Configure classicthesis for your needs here, e.g., remove "drafting" below
% in order to deactivate the time-stamp on the pages

\PassOptionsToPackage{
  drafting=false,           % print version information on the bottom of the pages
  tocaligned=false,         % the left column of the toc will be aligned (no indentation)
  dottedtoc=true,           % page numbers in ToC flushed right
  eulerchapternumbers=true, % use AMS Euler for chapter font (otherwise Palatino)
  linedheaders=false,       % chaper headers will have line above and beneath
  floatperchapter=true,     % numbering per chapter for all floats (i.e., Figure 1.1)
  eulermath=false,          % use awesome Euler fonts for mathematical formulae (pdfLaTeX)
  beramono=true,            % toggle a nice monospaced font (w/ bold)
  palatino=true,            % deactivate standard font for loading another one, see the last 
                            % section at the end of this file for suggestions
  style=arsclassica         % classicthesis, arsclassica
}{classicthesis}

%----------------------------------------------------------------------------------------
%	PERSONAL DATA AND USER AD-HOC COMMANDS
%----------------------------------------------------------------------------------------
%\newcommand{\myTitle}{Title\xspace}
\newcommand{\myTitle}{Methoden der Analyse von Sozialen Netzwerken\xspace}
\newcommand{\mySubtitle}{Bachelor Thesis\xspace}
\newcommand{\myDegree}{Bachelor of Science\xspace}
\newcommand{\myName}{Tanja \& Zast\xspace}
\newcommand{\myProf}{Prof. Dr.-Ing. Dr. h.c. Stefan Wesner\xspace}
\newcommand{\myOtherProf}{Dr. Dipl.-Inf. Lutz Schubert\xspace}
\newcommand{\myFaculty}{Faculty of Engineering, Computer Science and Psychology\xspace}
\newcommand{\myDepartment}{Institute of Information Resource Management\xspace}
\newcommand{\myUni}{Ulm University\xspace}
\newcommand{\myLocation}{Ulm\xspace}
\newcommand{\myTime}{April 2022\xspace}
\newcommand{\myVersion}{\classicthesis}

%----------------------------------------------------------------------------------------
%	SETUP & FINE-TUNING
%----------------------------------------------------------------------------------------
\providecommand{\mLyX}{L\kern-.1667em\lower.25em\hbox{Y}\kern-.125emX\@}
\newcommand{\ie}{i.\,e.}
\newcommand{\Ie}{I.\,e.}
\newcommand{\eg}{e.\,g.}
\newcommand{\Eg}{E.\,g.}


%----------------------------------------------------------------------------------------
%	PACKAGES
%----------------------------------------------------------------------------------------
\PassOptionsToPackage{ngerman,american}{babel} % change this to your language(s), main language last
% Spanish languages need extra options in order to work with this template
%\PassOptionsToPackage{spanish,es-lcroman}{babel}
    \usepackage{babel}

\usepackage{csquotes}
\PassOptionsToPackage{%
  %backend=biber,bibencoding=utf8, %instead of bibtex
  backend=bibtex8,bibencoding=ascii,%
  language=auto,%
  style=numeric-comp,%
  %style=authoryear-comp, % Author 1999, 2010
  %bibstyle=authoryear,dashed=false, % dashed: substitute rep. author with ---
  sorting=nyt, % name, year, title
  maxbibnames=10, % default: 3, et al.
  %backref=true,%
  natbib=true % natbib compatibility mode (\citep and \citet still work)
}{biblatex}
    \usepackage{biblatex}

\PassOptionsToPackage{fleqn}{amsmath}       % math environments and more by the AMS
  \usepackage{amsmath}

%----------------------------------------------------------------------------------------
%	GENERAL USEFUL PACKAGES
%----------------------------------------------------------------------------------------
\usepackage{graphicx} %
\usepackage{scrhack} % fix warnings when using KOMA with listings package
\usepackage{xspace} % to get the spacing after macros right



%\usepackage{pgfplots} % External TikZ/PGF support (thanks to Andreas Nautsch)
%\usetikzlibrary{external}
%\tikzexternalize[mode=list and make, prefix=ext-tikz/]


%----------------------------------------------------------------------------------------
%	TABLES, (SUB)FIGURES, AND CAPTIONS
%----------------------------------------------------------------------------------------
\usepackage{tabularx} % better tables
  \setlength{\extrarowheight}{3pt} % increase table row height
\newcommand{\tableheadline}[1]{\multicolumn{1}{l}{\spacedlowsmallcaps{#1}}}
\newcommand{\myfloatalign}{\centering} % to be used with each float for alignment
% \usepackage{subfig}
%\usepackage{todonotes}
\usepackage{pdflscape}
\usepackage{caption}
\usepackage{subcaption}
\usepackage{placeins}


%----------------------------------------------------------------------------------------
%	CODE LISTINGS
%----------------------------------------------------------------------------------------
\usepackage{listings}
%\lstset{emph={trueIndex,root},emphstyle=\color{BlueViolet}}%\underbar} % for special keywords
\lstset{language=[LaTeX]Tex,%C++,
  morekeywords={PassOptionsToPackage,selectlanguage},
  keywordstyle=\color{RoyalBlue},%\bfseries,
  basicstyle=\small\ttfamily,
  %identifierstyle=\color{NavyBlue},
  commentstyle=\color{Green}\ttfamily,
  stringstyle=\rmfamily,
  numbers=none,%left,%
  numberstyle=\scriptsize,%\tiny
  stepnumber=5,
  numbersep=8pt,
  showstringspaces=false,
  breaklines=true,
  %frameround=ftff,
  %frame=single,
  belowcaptionskip=.75\baselineskip
  %frame=L
}

%----------------------------------------------------------------------------------------
%	MAIN PACKAGE
%----------------------------------------------------------------------------------------
\usepackage{classicthesis}
\usepackage{algorithm}
\usepackage[noend]{algpseudocode}

%----------------------------------------------------------------------------------------
%	HYPERREFERENCES (hyperref should be called last)
%----------------------------------------------------------------------------------------
\hypersetup{%
  %draft, % hyperref's draft mode, for printing see below
  colorlinks=true, linktocpage=true, pdfstartpage=3, pdfstartview=FitV,%
  % uncomment the following line if you want to have black links (e.g., for printing)
  %colorlinks=false, linktocpage=false, pdfstartpage=3, pdfstartview=FitV, pdfborder={0 0 0},%
  breaklinks=true, pageanchor=true,%
  pdfpagemode=UseNone, %
  % pdfpagemode=UseOutlines,%
  plainpages=false, bookmarksnumbered, bookmarksopen=true, bookmarksopenlevel=1,%
  hypertexnames=true, pdfhighlight=/O,%nesting=true,%frenchlinks,%
  urlcolor=CTurl, linkcolor=CTurl, citecolor=CTurl, %pagecolor=RoyalBlue,%
  %urlcolor=Black, linkcolor=Black, citecolor=Black, %pagecolor=Black,%
  pdftitle={\myTitle},%
  pdfauthor={\textcopyright\ \myName, \myUni, \myFaculty},%
  pdfsubject={},%
  pdfkeywords={},%
  pdfcreator={pdfLaTeX},%
  pdfproducer={LaTeX with hyperref and classicthesis}%
}

%----------------------------------------------------------------------------------------
%	AUTOREFERENCES
%----------------------------------------------------------------------------------------
% There are some issues regarding autorefnames
% http://www.tex.ac.uk/cgi-bin/texfaq2html?label=latexwords
% you have to redefine the macros for the
% language you use, e.g., american, ngerman
% (as chosen when loading babel/AtBeginDocument)

\makeatletter
\@ifpackageloaded{babel}%
  {%
    \addto\extrasamerican{%
      \renewcommand*{\figureautorefname}{Figure}%
      \renewcommand*{\tableautorefname}{Table}%
      \renewcommand*{\partautorefname}{Part}%
      \renewcommand*{\chapterautorefname}{Chapter}%
      \renewcommand*{\sectionautorefname}{Section}%
      \renewcommand*{\subsectionautorefname}{Section}%
      \renewcommand*{\subsubsectionautorefname}{Section}%
    }%
    \addto\extrasngerman{%
      \renewcommand*{\paragraphautorefname}{Absatz}%
      \renewcommand*{\subparagraphautorefname}{Unterabsatz}%
      \renewcommand*{\footnoteautorefname}{Fu\"snote}%
      \renewcommand*{\FancyVerbLineautorefname}{Zeile}%
      \renewcommand*{\theoremautorefname}{Theorem}%
      \renewcommand*{\appendixautorefname}{Anhang}%
      \renewcommand*{\equationautorefname}{Gleichung}%
      \renewcommand*{\itemautorefname}{Punkt}%
    }%
      % Fix to getting autorefs for subfigures right (thanks to Belinda Vogt for changing the definition)
      \providecommand{\subfigureautorefname}{\figureautorefname}%
    }{\relax}
\makeatother
