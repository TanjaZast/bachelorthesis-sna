%*****************************************
\chapter{Fazit und Ausblick}\label{ch:fazit}
%*****************************************

Nachdem in dieser Arbeit ausführlich die Generierung und Analyse sozialer Netzwerk behandelt wurde, werden nun die wichtigsten Erkenntnisse zusammengeführt. \\
Die Analyse sozialer Netzwerke besteht aus vielen Faktoren. Es gibt zahlreiche Methoden um eine Analyse durchzuführen und viele charakteristische Merkmale, die bei dieser von Bedeutung sind, siehe Tabelle \ref{TableEigenschaften} und Tabelle \ref{TableEigenschaften2.0}. In dieser Arbeit wurde lediglich ein Teil davon betrachtet, was es bei der Analyse von sozialen Netzwerken noch zu untersuchen gäbe. Zentralitäten spielen bei der Analyse beispielsweise eine wichtige Rolle, da sie in direktem Zusammenhang mit Cliquen und Brücken stehen. Die Arbeit hat gezeigt, dass Cliquen höhere \textit{Grad-Zentralitäten} aufweisen, und hohe \textit{Zwischen-Zentralitäten} bedeuten, dass dieser Knoten relevant für Brücken zwischen den Teilgraphen ist. In dieser Arbeit wurde zudem gezeigt, dass die Betrachtung der Cliquen und Brücken bei der visuellen Interpretation sehr aufschlussreich ist und bereits Vermutungen entstehen lässt. \\

Bei der Analyse der Daten ist zudem deutlich geworden, dass es hilfreich ist, die Verteilungen dieser zu betrachten. Denn oftmals werden Graphen analysiert mit einer großen Menge an Knoten und Kanten. Hier bietet die Betrachtung der Verteilung eine gute Möglichkeit, einen Überblick der Zentralitäten zu bekommen und parallel, ohne den Plot dazu gesehen zu haben, die Visualisierung zu erahnen. 
Zwar sind die Verteilungen, die in dieser Arbeit erzielt wurden, nicht identisch gewesen doch konnten stets Parallelen zur gesuchten Poisson-Verteilung nachgewiesen werden.
Gleichzeitig ist in dem Zusammenhang mit Zentralitäten zu beachten, dass soziale Netzwerke unterschiedlichste Thematiken darstellen können. Alleine eine kurze Suche im Internet präsentiert unzählige unterschiedliche Netzwerke. Diese weisen visuell gesehen starke Unterschiede auf oder womöglich keine Ähnlichkeiten, doch haben diese dennoch eine Gemeinsamkeit, sie sind alle auf ihre Weise soziale Netzwerke und erfüllen dennoch die Kriterien dafür. Wie beispielsweise mehrere Teilgraphen, Cliquen, Brücken und zudem ähnliche Verteilungen der Zentralitäten. Zwar kann die Interpretation dieser, sehr unterschiedlich aussehender Netzwerke, schwierig sein, doch können die untersuchten Merkmale trotzdem ähnliche oder sogar identische Ergebnisse erzielen. \\

Anhand dieser Arbeit wurde ebenfalls ersichtlich, dass bereits kleine Optimierungen im Quelltext des Generators, die Generierung visuell ähnlicher Graphen ermöglicht. Denn sobald Graphen gleiche oder ähnliche visuelle Grundstrukturen aufweisen, folgen direkt auch starke Gleichheiten in der Verteilung der Zentralitäten. Diese Arbeit lässt einige Punkte offen, die durchaus noch weiter optimiert werden können. Beispielsweise die generierten Plots noch besser an existierende Graphen anpassen, um die Verteilung bestmöglich nachzustellen. 
Der Generator in dieser Arbeit bildet lediglich mehrere Cluster und verbindet diese miteinander, wobei es in \textit{sozialen Netzwerken} auch Knoten geben kann, die sich zwischen den Subgraphen befinden, siehe Abbildung \ref{fig:FacebookGraph}. Dies könnte durch eine weitere Methode im Quelltext nachgestellt werden. Zudem wäre ein weiterer interessanter Faktor die Dichte in den Cluster zu untersuchen und festzustellen, ob die Knoten sehr nah beieinander liegen oder weit voneinander entfernt sind. Dies würde sich wiederum auf die Werte der Zentralitäten und schließlich deren Verteilungen auswirken.
Auch die Untersuchung größere Datensätze, oder der Vergleich dieser wäre eine interessante Fortsetzung dieser Arbeit. Ein weiterer Ansatz für die Fortsetzung der Arbeit wäre die Untersuchung der zur Berechnung verwendeten Algorithmen. Der Frage diesbezüglich nachzugehen, ob die Algorithmen zur Berechnungen der Zentralitäten bereits optimiert sind oder ob nicht möglicherweise Verbesserungspotenzial besteht.  
Alle diese Ideen zeigen erneut, wie vielfältig soziale Netzwerke sind und die Analyse dieser ist und warum sie zahlreiche Wisschenschaftler*innen seit Jahren beschäftigt. Schließlich kann diese Arbeit damit beendet werden, dass es unglaublich vielzählige Methoden zu Analyse von Netzwerken gibt. Welche die geeignetste ist, ob es möglicherweise viel bessere gibt, die wir womöglich nicht betrachtet haben, lässt sich nicht in einem Satz zufriedenstellend beantworten. Es kommt auf die Anzahl der Kanten und Knoten an, aber auch auf die zu untersuchende Thematik.