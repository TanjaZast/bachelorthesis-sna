%************************************************
\chapter{Einleitung}\label{ch:einleitung}
%************************************************
Der Begriff \textit{soziales Netzwerk} oder auf Englisch \textit{social network} weckt seit vielen Jahrzehnten das Interesse zahlreicher Sozial- und Verhaltenswissenschaftler*innen \cite{SNAIntroduction}. Neben diesen, weckt es zudem das Interesse von unzähligen Unternehmen, um gezielter auf das Kundenverhalten einzugehen und dadurch den Gewinn zu maximieren \cite{CompanySNA}. Doch vor allem nicht zu vergessen, sind es heutzutage letztendlich die Nutzer*innen der Social Media-Plattformen wie Twitter, Facebook und Instagram, welche von diesem Begriff vor allem betroffen sind und auf diese Weise könnte die Liste noch lange fortgeführt werden.\\
Um jedoch eine klare Aussage zu treffen, spezialisieren sich vor allem Sozial- und Verhaltenswissenschaftler*innen, ebenso Unternehmen, auf die Analyse sozialer Netzwerke \cite{SNAIntroduction, CompanySNA}. Diese fokussieren sich weitestgehend auf Beziehungen zwischen sozialen Einheiten, sowie die Muster und Implikationen, welche diesen Beziehungen zugeschrieben werden \cite{networkPattern}.


\section{Zielsetzung}\label{sec:zielsetzung}
Wie ist ein \textit{soziales Netzwerk} definiert und wie kann eine Analyse dieses Netzwerks aussehen? Was zeichnet die einzelnen Methoden zur Analyse aus und welche gelten als besonders aussagekräftig? Ziel der Arbeit ist es, einen Generator für \textit{soziale Netzwerke} zu erstellen, welcher selbst generierte oder vorgegebene Testdaten visuell darstellt und zudem automatisch einige Methoden zur Analyse durchführt, beziehungsweise anwendet. Dafür muss zunächst ein Verständnis entwickelt werden, was ein \textit{soziales Netzwerk} auszeichnet und von zufälligen Netzwerken unterscheidet.
Diese Arbeit wird daher in zwei Bereiche unterteilt. Zum Einen in die Einführung von \textit{sozialen Netzwerken} und die Erörterung der verschiedenen Zentralitäten und Eigenschaften von Cliquen und Brücke, die Aufschluss darüber geben, wie die Einheiten miteinander verbunden sind, beziehungsweise zusammenhängen. Anschließend wird im zweiten Teil dieser Arbeit der Generator entwickelt, welcher Soziale Netzwerke so gut wie möglich nachstellt. Zum Schluss werden mehrere Analysen durchgeführt und die Verteilungen der Zentralitätswerte genauer betrachtet.\\
Diese Arbeit distanziert sich von dem Begriff \textit{social networking}, welcher bei Recherchen zahlreichst auftaucht, aber lediglich den Vorgang oder Zustand beschreibt, dass Menschen über soziale Netzwerke durch beispielsweise gemeinsame Interesse zueinanderfinden. 




%*****************************************
%*****************************************
%*****************************************
%*****************************************
%*****************************************
