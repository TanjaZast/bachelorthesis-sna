%*****************************************
\chapter{Fazit und Ausblick}\label{ch:fazit}
%*****************************************

Nachdem in dieser Arbeit ausführlich die Generierung und Analyse sozialer Netzwerk behandelt wurde, werden nun die wichtigsten Erkenntnisse zusammenführen. Die Analyse sozialer Netzwerke besteht aus vielen Faktoren. Es gibt zahlreiche Methoden um eine Analyse durchzuführen und viele charakteristische Merkmale, die bei dieser von Bedeutung sind. In dieser Arbeit wurde gezeigt, dass die Betrachtung der Cliquen und Brücken bei der visuellen Interpretation ausreicht. Bei der Analyse der Daten ist zudem deutlich geworden, dass es besser ist die Verteilungen dieser zu betrachten. Gleichzeitig ist zu beachten, dass soziale Netzwerke unterschiedlichste Thematiken darstellen. Alleine eine kurze Recherche im Internet präsentiert unzählige unterschiedliche Netzwerke. Doch haben diese eine Gemeinsamkeit, sie sind alle auf ihre weise typische soziale Netzwerke. Anhand dieser Arbeit wurde ebenfalls ersichtlich, dass bereits kleine Optimierungen im Code, die Generierung visuell ähnlicher Graphen ermöglicht. Sobald die Graphen gleiche visuelle Grundstrukturen aufweisen, folgen auch starke Gleichheiten in der Verteilung der Zentralitäten. Diese Arbeit lässt einige Punkte offen, die durchaus noch weiter optimiert werden können. Beispielsweise die generierten Plots noch besser an existierende Graphen anpassen, um die Verteilung bestmöglich nachzustellen. 
Unser Generator bildet lediglich kleine Subgraphen und verbindet diese, wobei aber es auch Knoten geben kann, die sich zwischen den Clustern befinden siehe \ref{fig:FacebookGraph}. Dies könnte durch eine weitere Methode im Code erzeugt werden. Zudem wäre ein weiterer interessanter Faktor die Dichte in diesen Clustern, ob die Knoten sehr nah beieinander liegen oder weit voneinander entfernt sind.
Auch größere Datensätze zu untersuchen und zu vergleichen wäre eine interessante Fortsetzung dieser Arbeit. Oder ebenfalls interessant, ob die Berechnungen der Zentralitäten bereits optimierte Algorithmen sind und ob nicht möglicherweise doch Optimierungspotenzial besteht.  
All diese Ideen zeigen erneut, wie vielfältig soziale Netzwerke und die Analyse sind und warum sie zahlreiche Wisschenschaftler*innen beschäftigt. Schließlich kann diese Arbeit damit beendet werden, dass es unglaublich vielzählige Methoden zu Analyse von Netzwerken gibt. Welche die geeignetste ist, lässt sich nicht in einem Satz formulieren. Es kommt auf Anzahlen von Kanten und Knoten an, aber auch auf die zu untersuchende Thematik.